% Options for packages loaded elsewhere
\PassOptionsToPackage{unicode}{hyperref}
\PassOptionsToPackage{hyphens}{url}
%
\documentclass[
]{article}
\usepackage{amsmath,amssymb}
\usepackage{iftex}
\ifPDFTeX
  \usepackage[T1]{fontenc}
  \usepackage[utf8]{inputenc}
  \usepackage{textcomp} % provide euro and other symbols
\else % if luatex or xetex
  \usepackage{unicode-math} % this also loads fontspec
  \defaultfontfeatures{Scale=MatchLowercase}
  \defaultfontfeatures[\rmfamily]{Ligatures=TeX,Scale=1}
\fi
\usepackage{lmodern}
\ifPDFTeX\else
  % xetex/luatex font selection
\fi
% Use upquote if available, for straight quotes in verbatim environments
\IfFileExists{upquote.sty}{\usepackage{upquote}}{}
\IfFileExists{microtype.sty}{% use microtype if available
  \usepackage[]{microtype}
  \UseMicrotypeSet[protrusion]{basicmath} % disable protrusion for tt fonts
}{}
\makeatletter
\@ifundefined{KOMAClassName}{% if non-KOMA class
  \IfFileExists{parskip.sty}{%
    \usepackage{parskip}
  }{% else
    \setlength{\parindent}{0pt}
    \setlength{\parskip}{6pt plus 2pt minus 1pt}}
}{% if KOMA class
  \KOMAoptions{parskip=half}}
\makeatother
\usepackage{xcolor}
\usepackage[margin=1in]{geometry}
\usepackage{graphicx}
\makeatletter
\def\maxwidth{\ifdim\Gin@nat@width>\linewidth\linewidth\else\Gin@nat@width\fi}
\def\maxheight{\ifdim\Gin@nat@height>\textheight\textheight\else\Gin@nat@height\fi}
\makeatother
% Scale images if necessary, so that they will not overflow the page
% margins by default, and it is still possible to overwrite the defaults
% using explicit options in \includegraphics[width, height, ...]{}
\setkeys{Gin}{width=\maxwidth,height=\maxheight,keepaspectratio}
% Set default figure placement to htbp
\makeatletter
\def\fps@figure{htbp}
\makeatother
\setlength{\emergencystretch}{3em} % prevent overfull lines
\providecommand{\tightlist}{%
  \setlength{\itemsep}{0pt}\setlength{\parskip}{0pt}}
\setcounter{secnumdepth}{-\maxdimen} % remove section numbering
\usepackage{NotesTemplateF}
\ifLuaTeX
  \usepackage{selnolig}  % disable illegal ligatures
\fi
\IfFileExists{bookmark.sty}{\usepackage{bookmark}}{\usepackage{hyperref}}
\IfFileExists{xurl.sty}{\usepackage{xurl}}{} % add URL line breaks if available
\urlstyle{same}
\hypersetup{
  pdftitle={Polynomials},
  pdfauthor={Aathreya Kadambi},
  hidelinks,
  pdfcreator={LaTeX via pandoc}}

\title{Polynomials}
\author{Aathreya Kadambi}
\date{February 23, 2024}

\begin{document}
\maketitle

I recently read a post from Terrance Tao's blog on the Yoneda Lemma
which revived my interest in polynomials. I've also been seeing
polynomials a lot in my algorithms class as we study FFTs. I really like
polynomials, so I decided to dedicate this blog post to a thorough
investigation of them. I chose to use an R markdown notebook because in
my opinion it has the cleanest integration between LaTeX, HTML, and a
powerful programming language I can use for implementations.

\hypertarget{introduction}{%
\subsection{Introduction}\label{introduction}}

To start, I'll summarize a lot of the notation and results we'll need.
This can be freely skipped if you're confident in your background on
polynomials, and proofs are found in the appendix.

\textbf{Definition (Integer Polynomial).} A \emph{polynomial} is an
expression of the form
\(f(x) = a_0 + a_1 x + a_2 x^2 + \dots + a_n x^n\) for some nonnegative
integer \(n\) and integers \(a_0\), \(a_1\), \ldots{} which are called
\emph{coefficients} of \(f\).

Some may be more familiar with a more general definition of polynomial
(like the one used in Lang's \emph{Algebra}), where instead of having
integer coefficients we can have coefficients in any ring \(R\).

One big idea about polynomials is how to represent them. There is a very
useful theorem which we call Lagrange interpolation. I will give the
statement of this theorem from the original place I learned it, a note
by Yufei Zhao:

\begin{theorem}
  **Theorem (Lagrange Interpolation).** If $(x_1,y_1), \dots, (x_n,y_n)$ are points in the plane with distinct $x$-coordinates, then there exists a unique polynomial $P(x)$ of degree at most $n-1$ passing through these points, and it is given by the expression
  $$P(x) = \sum_{i=1}^n y_i \prod_{j\neq i} \dfrac{x - x_j}{x_i - x_j}.$$
\end{theorem}

Notice also that given values \(\{x_1,\dots, x_n\}\), a polynomial \(P\)
gives us a collection of points \((x_1,P(x_1)), \dots, (x_n, P(x_n))\).
As such, for any selection of points \(\{x_1,\dots, x_n\}\), there is a
correspondence between polynomials \(P\) of degree at most \(n-1\) and
``\(y\)-values'' \(y_1,\dots, y_n\).

\hypertarget{fast-fourier-transform}{%
\subsection{Fast Fourier Transform}\label{fast-fourier-transform}}

\hypertarget{yoneda-lemma}{%
\subsection{Yoneda Lemma}\label{yoneda-lemma}}

\hypertarget{relating-the-pictures}{%
\subsection{Relating the Pictures}\label{relating-the-pictures}}

\hypertarget{geometric-insights}{%
\subsection{Geometric Insights}\label{geometric-insights}}

\hypertarget{power-series-and-generating-functions}{%
\subsection{Power Series and Generating
Functions}\label{power-series-and-generating-functions}}

\hypertarget{appendix-proofs}{%
\subsection{Appendix: Proofs}\label{appendix-proofs}}

\end{document}
